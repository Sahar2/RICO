INFORMAL DISCUSSION

Since RICO is able to compute derivatives symbolically the family of Householder methods are available. \todo{reference other than Wikipedia}
Halley's Method is used in RICO. The reason is that it seems to provide a good performance and does not suffer from the tendency of Newtons to shoot beyond the bounded interval when a relatively flat spot of the objective function is encountered.

A problem shared by Newton and Halley is that there are initial points from with they fail to locate existing roots. They are therefore unreliable in discrediting the presence of a root, but very fast in determining the presence of a root.

Suppose that Halley's Method were unleashed on a bounded interval first. The cost is the number of function evaluations and the computation of the first and second derivative and their evaluations. At this point I have no way to decide when Newton or Halley is more appropriate. They share a deficiency. The derivative computations are symbolic and therefore relatively fast. The function evaluations are based on symbolic description and potentially fast depending on complexity of the description. The problem is that we're not so much interested in roots as determining that intervals do not contain roots. Finding a root only allows us to surround it with a tiny guard interval and return to searching the two flanking intervals. We usually don't know how many roots there are so we can't just look for roots.

I think Halley could be good at the tail end of the interpolation subroutine. 
